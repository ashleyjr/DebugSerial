
\section{Terminal}

Terminal mode is the most basic data handling tool available in \ds.
An keyboard input the can be taken as an ASCII value if piped down the chosen serial port.
Returning data is translated from raw form to the equivalent ASCII character and printed in the terminal.
As an example connecting the transmit and receive end of a serial port together will create a text editor like interface that will filter any non-ASCII keys from being printed.
Some ASCII characters cannot be visualised easily within a terminal environment so these are replaced by the name of the representation encase by angle brackets; <bell>, <alt>, etc.
