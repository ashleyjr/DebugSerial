\documentclass[twoside,12pt]{article}

%\usepackage[nodayofweek]{datetime}


\usepackage{multirow}
\usepackage[colorinlistoftodos,disable]{todonotes}
\usepackage[nodayofweek]{datetime}
\usepackage{listings}
\usepackage{graphicx}
\usepackage{titlesec}
\newcommand{\sectionbreak}{\cleardoublepage}
\usepackage[all]{background}
\usepackage{lipsum}
\usepackage{tikz}
\usetikzlibrary{calc}
\usepackage{changepage}
\usepackage{subfigure}
\graphicspath{{Figures/}}

\usepackage{fancyhdr, graphicx}
\renewcommand{\headrulewidth}{0pt}

\fancyfoot[C]{
	\setlength{\unitlength}{1mm}
	\begin{picture}(0,10)
	\put(-70,40){\includegraphics[width=14cm]{cover.jpg}}
	\end{picture}
}


\pagestyle{plain}



\author{
Ashey J. Robinson\\ \texttt{ashley181291@gmail.com}
 \\}

\title{DebugSerial\\ User Manual}

\begin{document}
\maketitle
\thispagestyle{fancy}


\input{definitions.tex}%import some definitions
\clearpage
%\listoftodos
%\todo[inline]{Run a spell check on all files}
%\begin{abstract}
%Abstract
%\end{abstract}

\newpage
\cleardoublepage
\begin{abstract}

DebugSerial is a tool for debugging embedded software using a simple serial connection.
Only a UART with eight bits of data and a single stop bit is permitted but varying baud rates are allowed.
Motivation for this tool is to increase design potential when only simple hardware is available.
UART protocol is slow but is possible to setup real-time graphs and other interesting analysis techniques.
This software relieves the embedded systems engineer from creating higher level protocols for sending data which can be time consuming when managing both ends of the connections.
Protocols are defined and a rich GUI is provided so debugging using serial is simple from hardware all the way to visualisation.

\end{abstract}
\cleardoublepage
\tableofcontents
\cleardoublepage
\section{Introduction}

\subsection{Hardware}

\subsection{Getting Started}
\ds is invoked as a python script from a terminal window.
\ds will immediately search the environment for all available serial ports.
Failure to find any serial ports will cause \ds to exit.
If only one serial port is available this will be connected to automatically.
When more than one serial port is found a choice will presented.
User choices persist between program executions.

When a serial port is selected a baud rate must be chosen.
Any value can be taken  but this may be limited by chosen hardware.



\end{document}

